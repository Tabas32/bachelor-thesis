% !TEX root = ../thesis.tex

\chapter{Tvorba datasetu}
\label{tvorba_datasetu}

Najdôležitejšou časťou pri vytváraní nášho systému je tvorba datasetu.
Podľa toho aké dobré dáta zozbierame, také výsledky sa nám podarí získať.
Náš dataset bude tvorený dvomi druhmi dát, a to obrazovými a zvukovými.

\section{Zvukové dáta}
V predošlej kapitole sme vysvetlili prečo a ako je potrebné upraviť hudobné dáta, z ktorých sa bude náš model učiť.
V tejto podkapitole opíšeme ako bude vyzerať zvuková časť nášho datasetu.

Prvým krokom je výber žánru skladieb, ktoré budeme zbierať.
V súčasnosti existuje nesmierne množstvo interpretov a skladieb z ktorých si môžeme vybrať.
No nie každý hudobný žáner dokáže podnietiť našu predstavivosť.
Pre náš projekt budú najvhodnejšie skladby klasickej hudby, nakoľko na rozdiel od populárnej modernej hudby, klasická hudba dáva väčší dôraz na emócie.

Vzorky datasetu by mali mať konštantnú veľkosť.
Preto je nevyhnutné upraviť skladby tak aby tomu vyhovovali.

V závislosti od dĺžky a rôznorodosti jednotlivých skladieb vyberieme z každej niekoľko vzoriek s dĺžkou 25 sekúnd.
Táto dĺžka je zvolená preto aby mal poslucháč dostatok času na vybavenie obrazu.
Pomocou programu Audacity ďalej upravíme každú vzorku pomocou efektu Normalizovať.
Takto dosiahneme podobnú hlasitosť, bez ohľadu na spôsob nahrávania daných vzoriek.
Taktiež zmeníme vzorkovaciu frekvenciu na 22 050Hz, aby naša analýza bola súvislá.

Pre hlavnú analýzu vytvoríme skript v programovacom jazyku Python.
Tento jazyk sme vybrali kvôli možstvu modulov, ktoré existujú a pomôžu nám pri tvorbe systému.
Pre hudobnú analýzu použijeme modul Librosa, ktorý obsahuje množstvo algoritmov vhodných pre získanie vlastností akustických signálov, ktoré sme uviedli v predošlej kapitole.

Vybrali sme päť akustických vlastností, ktoré budú reprezentovať naše skladby, a to mel-frekvenčné kepstrálne koeficienty, tempo, spektrálne centroidy, počet nulových priechodov a mieru týchto priechodov.
Každá z týchto vlastností nesie užitočnú informáciu potrebnú pre našu úlohu.
Napríklad mel-frekvenčné kepstrálne koeficienty poskytujú informáciu o tom aký tón a farbu má skladba v danom časovom úseku.

Touto analýzou sme úspešne skrátili vstupný vektor.
Ak by sme chceli vložiť 25 sekundové vzorky skladieb do neurónovej siete potrebovali by sme viac ako 500 000 vstupných kanálov.
Náš dataset obsahuje pre každú vzorku približne 700 reálnych hodnôt.

\section{Obrazové dáta}
Tak ako hudobné dáta tak aj obrázky, ktoré bude náš dataset obsahovať musia mať nejakú koherentnú formu.

Prvým modelom, ktorý budeme vytvárať bude neurónová sieť, ktorá dokáže priradiť k skladbe farbu.
Pre tento klasifikátor ako výstup použijeme jedno číslo, ktoré bude reprezentovať farbu v našej farebnej palete.
Takto sme sa rozhodli z dôvodu, že pri vytváraní datasetu bude jednoduchšie ak k jednotlivým skladbám priradíme farbu z vopred danej palety.

Pre druhý a náš hlavný model, budeme ukladať obrázky vo formáte png a každý obrázok bude uložený vo veľkosti 32x32 pixlov.

% !TEX root = ../thesis.tex

\chapter{Tvorba datasetu}
\label{tvorba_datasetu}

Najdôležitejšou časťou pri vytváraní nášho systému je tvorba datasetu.
Podľa toho aké dobré dáta zozbierame, také výsledky sa nám podarí získať.
Náš dataset bude tvorený dvomi druhmi dát, a to obrazovými a zvukovými.

\section{Zvukové dáta}
V predošlej kapitole sme vysvetlili prečo a ako je potrebné upraviť hudobné dáta, z ktorých sa bude náš model učiť.
V tejto podkapitole opíšeme ako bude vyzerať zvuková časť nášho datasetu.

Prvým krokom je výber žánru skladieb, ktoré budeme zbierať.
V súčasnosti existuje nesmierne množstvo interpretov a skladieb z ktorých si môžeme vybrať.
No nie každý hudobný žáner dokáže podnietiť našu predstavivosť.
Pre náš projekt budú najvhodnejšie skladby klasickej hudby, nakoľko na rozdiel od populárnej modernej hudby, klasická hudba dáva väčší dôraz na emócie.

Vzorky datasetu by mali mať konštantnú veľkosť.
Preto je nevyhnutné upraviť skladby tak aby tomu vyhovovali.

V závislosti od dĺžky a rôznorodosti jednotlivých skladieb vyberieme z každej niekoľko vzoriek s dĺžkou 25 sekúnd.
Táto dĺžka je zvolená preto aby mal poslucháč dostatok času na vybavenie obrazu.
Pomocou programu Audacity ďalej upravíme každú vzorku pomocou efektu Normalizovať.
Takto dosiahneme podobnú hlasitosť, bez ohľadu na spôsob nahrávania daných vzoriek.
Taktiež zmeníme vzorkovaciu frekvenciu na 22 050Hz, aby naša analýza bola súvislá.

Pre hlavnú analýzu vytvoríme skript v programovacom jazyku Python.
Tento jazyk sme vybrali kvôli možstvu modulov, ktoré existujú a pomôžu nám pri tvorbe systému.
Pre hudobnú analýzu použijeme modul Librosa, ktorý obsahuje množstvo algoritmov vhodných pre získanie vlastností akustických signálov, ktoré sme uviedli v predošlej kapitole.

Vybrali sme päť akustických vlastností, ktoré budú reprezentovať naše skladby, a to mel-frekvenčné kepstrálne koeficienty, tempo, spektrálne centroidy, počet nulových priechodov a mieru týchto priechodov.
Každá z týchto vlastností nesie užitočnú informáciu potrebnú pre našu úlohu.
Napríklad mel-frekvenčné kepstrálne koeficienty poskytujú informáciu o tom aký tón a farbu má skladba v danom časovom úseku.

Touto analýzou sme úspešne skrátili vstupný vektor.
Ak by sme chceli vložiť 25 sekundové vzorky skladieb do neurónovej siete celkový počet vstupných kanálov by bol \( 25 \times 22050 = 551250\).
Náš dataset obsahuje pre každú vzorku 549 hodnôt z oboru reálnych čísel.
Mel-frekvenčné kepstrálne koeficienty tvoria najväčšiu časť.
\begin{minted}{python}
S = librosa.feature.melspectrogram(
	y=y, sr=sr, n_mels=128, hop_length=22050, n_fft=22050)
log_S = librosa.logamplitude(S, ref_power=np.max)
mfcc = librosa.feature.mfcc(S=log_S, sr=sr, n_mfcc=20)
mfcc_1d_vector = mfcc.flatten()
\end{minted}
Tie sú vypočítané na základe mel-spektrogramu so vzorkovaním 22050, čo vytvára 26 hodnôt.
Počet mfcc je 20, celkovo tak dostávame \( 20 \times 26 = 520\) čísel.
Tempo, aritmetický priemer spektrálnych centroidov a celkový počet nulových priechodov sú ďalšie 3 hodnoty datasetu.
Posledných 26 hodnôt tvoria miery nulových prechodov, merané so skokom 22050.
\begin{minted}{python}
zcr = librosa.feature.zero_crossing_rate(
	y, frame_length=22050, hop_length=22050).flatten()
\end{minted}


\section{Obrazové dáta}
Tak ako hudobné dáta tak aj obrázky, ktoré bude náš dataset obsahovať musia mať nejakú koherentnú formu.

Prvým modelom, ktorý budeme vytvárať bude neurónová sieť, ktorá dokáže priradiť k skladbe farbu.
Pre tento klasifikátor ako výstup použijeme jedno číslo, ktoré bude reprezentovať farbu v našej farebnej palete.
Takto sme sa rozhodli z dôvodu, že pri vytváraní datasetu bude jednoduchšie ak k jednotlivým skladbám priradíme farbu z vopred danej palety.

Pre druhý a náš hlavný model, budeme ukladať obrázky vo formáte png a každý obrázok bude uložený vo veľkosti 32x32 pixlov.

\section{Zber dát}
V tejto časti opíšeme spôsob ako sme zbierali dáta pre náš dataset.
Zber údajov je veľmi podstatná časť projektu.
Preto je nevyhnutné nebrať tento proces na ľahkú váhu.
Vytváranie vlastností (features) a štítkov (labels) môže byť úplne manuálne alebo z časti automatizované.
Zber dát manuálne môže byť veľmi pracný a zdĺhavý, preto je vhodné mať nejaký nástroj pre túto činnosť.
Tento nástroj by nás oslobodil od množstva repetitívnej práce.
Ďalšou výhodou pre vytváranie datasetu na jednom mieste je zapojenie viacerých ľudí do procesu.
Viac ľudí prináša variabilitu do našich dát a umožňuje vytvorenie robustnejšieho riešenia.

Z týchto dôvodov sme vytvorili webovú aplikáciu, ktorá slúži na zber údajov.
V našom datasete sú okrem akustických vlastností skladieb, ktoré dokážeme získať relatívne jednoducho, aj údaje, ktoré nedokážeme získať pomocou algoritmov.
Presne pre získanie týchto dát sme vytvorili formulár určený na ich ukladanie.
Konkrétne ide o farbu skladby, emóciu a obrázok, toho čo skladba predstavuje.
Farbu je možné vybrať z palety pätnástich základných farieb.
Pre emóciu možno vybrať jednu z piatich skupín emócií.
Jednotlivé skupiny možno opísať ako vášnivé emócie, veselé, dojímavé, vtipné a agresívne.
Najdôležitejšou informáciou, ktorú potrebujeme získať je ale obrázok.
Do nášho datasetu potrebujeme dva druhy obrázkov, a to jednoduchú skicu a detailný obrázok.
Pre jednouché stvárnenie skladby v podobe skice sme do formuláru pridali editor na nakreslenie obrázku.
Kvôli jednoduchosti je možné požiť len čiernu farbu.
V datasete potrebujeme obrázky o veľkosti 32x32, lenže kreslenie takýchto obrázkov by bolo nepraktické.
Preto sme plátnu na kreslenie dali veľkosť 320x320 a zväčšili sme hrúbku štetca.
Takto budeme môcť kresliť pohodlne a zároveň potom zmenšiť obrázky bez prílišnej straty.
Druhý typ obrázku získavame pomocou url adresy.
Detailnejšie obrázky vyhľadáme na internete a pomocou nášho nástroja uložíme adresu.
Databáza takto nebude musieť ukladať surové dáta a my budeme môcť obrázok jednoducho vyhľadať a stiahnuť.
Keďže naša aplikácia verejne dostupná, pribudla nám nová úloha, a to kontrola.
Skôr ako budeme môcť využiť nazbierané dáta, musíme ich skontrolovať či neobsahujú nevhodný obsah a či sú vierohodné.

Pomocou tejto webovej aplikácie sme zozbierali potrebné údaje, ktoré je nutné už len stiahnuť a jednoducho upraviť pre vstup do modelov neurónových sietí.

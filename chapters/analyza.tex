% !TEX root = ../thesis.tex

\chapter{Syntéza dát}
\label{synteza_dat}

Už od čias pred počítačmi ľudia využívali abstrakciu skutočných dát v podobe čísel a matematiky. S vývojom výpočtovej techniky prišli aj nové spôsoby zmeny jedného typu informácií na iný. Dnes existuje mnoho systémov určených na tento proces.

\section{Zmena textu na reč}
Už v osemdesiatych rokoch dvadsiateho storočia, keď Steve Jobs predstavil nový Macintosh, počítač vedel hovoriť.
Systémy, ktoré používajú zmenu textu na reč sú dnes bežná vec, a preto je ťažké predstaviť si svet bez nich \cite{text2speech}.
Aj keď sú zaužívané stále existuje priestor na zlepšenie. Hlas starého Macintosha bol zreteľne umelý, no dnes existujú programy, ktoré dokážu simulovať ľudskú reč takmer na nerozpoznanie od živých ľudí.
Tieto syntetizátory našli svoje využitie napríklad v telekomunikáciách.
Primitívne úlohy vykonávané cez telefón sú zverené počítačom.
Ľudia si už zvykli, že keď volajú niekam aby si niečo vybavili je normálne ak sa im ozve stroj.
Syntetizátory našli svoje využitie aj vo vzdelávaní.
Učenie jazykov z pohodlia domova je možné aj vďaka tomu, že počuť ako sa slovo vyslovuje môžeme bez prítomnosti skúseného rečníka či cudzinca.
Zrakovo postihnutý práve vďaka technológií zmeny textu na reč môžu využívať prístroje, ako napríklad počítače, telefóny a iné,  bez ktorých sa dnes už nezaobídeme.
Nie len slepím, ale aj nemým a inak telesne postihnutým pomáhajú čítačky textu každý deň. Svoje využitie našli v mnohých oblastiach od vedy a výskumu až po zábavný priemysel.

\section{Prevod reči na text}
Pre ľudí veľmi jednoduchá úloha, rozpoznanie reči, je pre číselné systémy netriviálna záležitosť.
Rozpoznanie nám známych rečových úkazov zo zvukovej vlny je takmer nemožné. Fourierové transformácie a úprava dát nám dávajú šancu na extrakciu vhodných informácií, ktoré sa stali vhodným nástrojom na detekciu slov \cite{speechRecognition}.
Systém schopný prevodu hovorenej reči na text je nevyhnutnosťou v prípadoch, keď človek nemôže alebo nedokáže použiť klávesnicu či iní mechanický vstup.
Osobný asistenti ako Cortana alebo Ok Google by bez týchto syntetizátorov nedosiahli takej popularity.
Prevodníky reči na text majú veľký vplyv na to ako pracujeme s našimi zariadeniami. Funkcie ako preklad z jedného jazyka do druhého v reálnom čase sa zavádzajú do programov určených na komunikáciu a zmenšujú tak priepasť medzi ľuďmi rôznych národností.
To by nebolo možné ak by jadrá týchto programov nestáli na technológiách prevodu reči na text a textu na reč.

\section{Rozpoznávanie obsahu dát}
S úlohou syntézy informácií súvisí aj problém reprezentovania významu dát.
Naučiť počítače rozpoznávať čo sa nachádza na obrázku je dnes silno skúmaná oblasť.
Mnohé automobilové spoločnosti sa snažia vytvoriť samo jazdiace vozidlá, ktoré dokážu spozorovať prekážky okolo seba a zareagovať rýchlejšia ako by to dokázal ktorýkoľvek človek.
Vďaka najnovším metodikám, ako napríklad využite kovolučných neurónových sietí,  ľudia vytvorili programy, ktoré dokážu rozpoznať čo sa nachádza na obrázkoch a rozoznať jednotlivé objekty \cite{tensorFlow}.
Takéto technológie sa dajú využiť napríklad pri vyhľadávaní. Spoločnosti ako Google ponúkajú cloudové služby stvorené presne na tieto účely \cite{googleCloud}.

Neurónové siete dokážu analyzovať obrázky na úrovniach aké doposiaľ neboli možné.
V roku 2016 výskumníci z Montrealu a Toronta vytvorili model, ktorý dokázal analyzovať obrázky a vytvoriť textový popis týkajúci sa obsahu obrázku \cite{imageCaption}.
V spojení s čítačkou textu by sme takto mohli ešte viac uľahčiť prístup k informáciám aj pre zdravotne postihnutých ale aj využiť takúto technológiu na ešte lepšie výsledky.
Detekcia tvári sa zavádza aj do mobilných telefónov a využíva sa na odomykanie uzamknutej obrazovky. Facebook vytvára systém, ktorý by využil rozpoznanie tváre a využil tieto dáta ako heslo pre používateľa \cite{facebook}.
Analýza obrazu ale neostáva len pri rozpoznávaní objektov. Inteligentné systémy dokážu rozpoznať rôzne vlastnosti obrazu.
Google vytvoril systém, ktorý nazval Deep dream. Deep dream dokáže rozpoznať vlastnosti obrazu a upraviť pôvodný obrázok pridaním vrstiev, ktoré majú podobné vlastnosti.
Vytvorené obrázky potom vyzerajú ako halucinácie.
Mnohé iné webové aplikácie zase využívajú siete, ktoré sa naučia ako rozpoznať štýl obrazu a vďaka tomu dokážu preniesť štýl na úplne iný obrázok. Vznikajú tak zaujímavé filtre na úpravu obrázkov a fotiek. 

\section{Generovanie obrázkov}
V posledných rokoch systémy strojového učenia preukazujú výsledky v generovaní nových dát.
Syntéza textu na obraz je jednou z najnovších prác v tomto odvetví \cite{text2image}.
Nakoľko ide o experiment, tak komerčné využitie ešte neexistuje. Ale pri zlepšení tejto technológie sa môže vytvoriť obrovský potenciál.
Takéto výskumy prebiehajú na celom svete a na prechod na trh určite nebudeme dlho čakať.
Umelo generované obrázky dokážeme vďaka toku ako sú vytvorené ďalej upraviť.
Nakoľko sú to obrázky vytvorené pomocou matematického modelu vieme upravovať obsah veľmi jednoducho.
Príkladom sú generované obrázky ľudí, v ktorých sa dá upravovať napríklad to či sa osoba usmieva alebo nie, či má okuliare alebo mnoho iných vlastnosí \cite{DCGAN}.

\chapter{Teoretický základ}
\label{teoreticky_zaklad}

Ako už bolo avizované naša práca sa zaoberá neurónovými sieťami. Preto v tejto kapytole zhrnieme základné pojmi a problémi, pred ktorými stojíme.

Algoritmus, ktorý sa učí z dát, poskituje dobé výsledky len vtedy ak má dobré dáta, z ktorých sa učí. 
Náš model sa bude učiť z hudobných dát.
Zvukové analógové sygnály uložené v digitálnej forme sú príliš rozsiahle nato aby sme ich mohli využiť ako vstup do neurónovej siete.
Jedinou možnosťou je úprava takýchto dát na jednoduchšiu formu.

\section{Hudobné dáta}
Ľudia vnímajú hudbu na rôznych úrovňach.
Rôzne vlastnosti harmonických zvukov vyvolávajú u nás rozličné emócie.
Napríklad kým vzrušenie blízko súvisí s tempom, tak výška tónov a hlasitosť skôr určujú náladu skladby \cite{emotionExtraction}.
To ako vnímame hudbu má preto veľký pliv na našu predstavivosť.

Hudobné dáta sú ľahko dostupné na internete. Čo sa týka hudby existuje niekoľko zdrojov pre informácie \cite{musicMining}:
\begin{itemize}
	\item Hudobné metadáta,
	\item Akustické vlastnosti,
	\item Slová,
	\item Hudobná kritika,
	\item MIDI súbory,
	\item Hudobné skóre.
\end{itemize}
Rôzne typy informácií majú využitie v rôznych prípadoch.
Nás budú zaujímať akustické vlastnosti, medzi ktoré patrí energia, rytmus, časové a spektrálne vlastnosti \cite{emotionExtraction}.
Pre každú z týchto sád vlastností existuje niekoľko vlastnosí, ktoré sa dajú extrahovať známimi algoritmami a programmi.
V tabuľke 2.1 sme uviedli zopár významných údajov, ktoré by sme mohli využiť pre trénovanie naších modelov.
Tieto údaje opisujú hudobnú informáciu a majú značne menšiu veľkosť ako skladba uložená vo formáte mp3 alebo wav.

\begin{table}[]
\centering
\caption{Akustické vlastnosti zvukových sygnálov}
\label{features_table}
\begin{tabular}{l p{10cm}}
\hline
\textbf{Sada vlastností} & \textbf{Extrahovateľné vlastnosti}                                                                                                                                                                             \\ \hline
Energia                  & Dynamická hlasitosť, výkon zvuku, celková hlasitosť, špecifické koeficienty citlivosti na hlasitosť                                                                  \\
Rytmus                   & Diagram úderov, vzor rytmu, histogram rytmu a tempo, sila rytmu, pravidelnosť rytmu, jasnosť rytmu, priemerná frekvencia nástupu, priemerné tempo                  \\
Časové vlastnosti        & Nulové priechody, logaritmus času útoku                                                                                                                                                                        \\
Spektrálne vlastnosti    & Spektrálne centroidy, spektrálne vyhodnocovanie, spektrálny tok, spektrálne merania rovinnosti, spektrálne hrudkové faktory, mel-frekvenčné kepstrálne koeficienty \\
Harmónia                 & Jasnosť kľúča, hudobný režim, harmonická zmena, diagram stúpania                                                                                                     \\ \hline
\end{tabular}
\end{table}

\section{Neurónové siete}
Tak ako mozog, neurónová sieť je spojenie viacerých neurónov do siete.
Neurón na vstupe príma informácie, ktoré pozmení a pošle na výstup, čo môže byť vstup ďalšieho neurónu.

Formálny neurón alebo perceptrón (obrázok 2.1) má na vstupe \(n\) aktivít \cite{NNKvasnicka}.
Označme vstup ako vektor \(x = (x_1, x_2, x_3, ... x_n)^T\), kde \(T\) označuje transponovaný vektor.

\begin{figure}[!ht] 
	\centering 
	\includegraphics[width=.6\textwidth]{figures/perceptron} 
	\caption{Model formálneho neurónu.} 
\end{figure}

Všetky vstupné kanály majú svoju váhu, označme preto váhy vstupov ako vektor \(w = (w_1, w_2, w_3, ... w_n)^T\).
Celkový vstup perceptrónu sa potom vypočíta ako súčet súčinu vektorov \(x\) a \(w\), a prahu excitácie \(\Theta\).
Pre získanie výstupu vstupy prejdú aktivačnou funkciou.
Pre výstup \(y\) potom platí \[y = f(w^Tx + \Theta)\]

Neuróny v sieťach sa spájajú do vrstiev. Nech je vrstva tvorená \(m\) neurónmi a každý neurón má \(n\) vstupných kanálov.
Označme váhu j-teho vstupného kanála do i-teho neurónu ako \(w_{ij}\).
Potom môžme vytvoriť váhovú maticu
\[
W = \begin{bmatrix}
		w_{11} & w_{12} & \dots & w_{1n} \\
		w_{21} & w_{22} & \dots & w_{2n} \\
		{...}							\\
		w_{m1} & w_{m2} & \dots & w_{mn}
     \end{bmatrix}
\]

Výstup takejto vrstvy pomot bude \(y = (y_1, y_2, y_3, ... y_m)^T\), ktorý dostanem ako \[y = Wx\]
Tento výstup potom slúži ako vstup do dalšej vrstvy, ktorej celkový počet vstupných kanálov je rovný \(m\).

Po prechode celou sieťou sa vypočíta strata s ohľadom na požadovaný výstup.
Trénovanie takéhoto modelu je potom minimalizačný problém kedy sa upravujú hodnoty váhových matíc a prahov aktivácie tak aby bola strata čo najmenšia.

Neurónové siete sú výborné klasifikátory ale dajú sa použiť aj na generovanie nových dát.
Existuje niekoľko druhov sietí, ktoré dokážu generovať obrázky.
Pre nás zaujímavé budú VAE, CPPN a GAN neurónové siete. 

\section{Variational Autoencoders}
VAEs alebo variational autoencoders vznikli upravením neurónových sietí známych ako autoencoders \cite{VAE}.
Tie fungujú na princípe hrdla. Ide vlastne o dve siete, kde jedna sieť zakóduje vstupnú informáciu na oveľa jednoduchší vektor a druhá sieť tento vektor dokáže rozkódovať opäť na pôvodnú informáciu.
Takéto neurónové siete sa dajú využiť napríklad pri kompresií dát ale nedokážu generovať nové dáta.
VAEs negenerujú vektor skutočných dát, VAEs generujú dva vektory.
Vektor stredných hodnôt a vektor smerodajnej odchýlky.
Tieto vektory sa potom spoja do jedného vektora, ktorý je vstupom do dekódera.
Takto VAE dokážu generovať dáta, ktoré sú podobné ale predsa rozdielne od vstupných reálnych dát.
Nevýhodou takýchto sietí je že ak sa použijú na generovanie obrázkov, tak generované obrázky sú rozmazané práve kvôli strate pri kompresií, ktorú VAEs vytvárajú.

\section{Generative Adversarial Networks}
GANs alebo generative adversarial networks, pôvodne navrhnuté Ianom Goodfellowom, fungujú na princípe konkurencie medzi dvoma sieťami \cite{GAN}.
Prvá sieť, generátor, sa snaží generovať čo najrealistickejšie dáta zo šumu. Druhá sieť, diskriminátor,  rozpoznáva či sú vstupy reálne alebo vygenerované.
Tieto siete sa tak snažia poraziť jedna druhú. GANs sú známe generovaním takmer realistických obrázkov.
No nevýhodou je, že sa veľmi ťažko trénujú. Je viacero situácií, ktoré môžu nastať. Môže sa stať, že generátor nájde systém ako oklamať diskriminátor a generuje len jeden druh informácií.
Alebo diskriminátor sa stane tak dobrým v rozpoznávaní skutočnosti, že vyhodnotí všetky generované dáta za neskutočné a tak sa generátor nebude môcť zlepšiť.

GANs zaznamenali niekoľko vylepšení a využití v rôznych oblastiach. Pri generovaní obrázkov sa osvedčilo využitie konvolučných vrstiev \cite{DCGAN}.
Takéto siete dokážu generovať oveľa kvalitnejšie obrázky aj keď stále nie vo veľkom rozlíšení.
Ďalším vylepšením, práve v tomto probléme bolo využitie viacerých GAN sietí \cite{stackGAN}. Nakopenie viacerých neurónových sietí má za následok lepšiu kvalitu generovaných obrázkov.
Niekoľko vrstiev dokáže zväčšiť obrázok s rozmermi 64x64 na rozmery 256x256 a výrazne upraviť kvalitu. A v neposlednom rade je aj výskum podmienených GANs.
Tie vznikajú pridaním vlastností na vstup, pričom sa takto dá viac, či menej ovplyvniť výstup \cite{conGAN}.
Práca z Univerzity v Michigane využíva práve podmienené GANs. V tejto práci využili neurónovú sieť na syntézu textu na obraz.
Z datasetu vtáctva a kvetín vytvorili model, ktorý dokáže generovať obraz podľa popisu \cite{text2image}.
S dostatočne veľkými zdrojmi by mohol mať tento prístup veľké využitie a mohol by zmeniť mnoho systémov. 

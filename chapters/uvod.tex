% !TEX root = ../thesis.tex

\chaptermark{Úvod}
\addcontentsline{toc}{chapter}{Úvod}

\chapter*{Úvod}

Systémy, ktoré sa dokážu učiť z dát sú už dnes prístupné verejnosti. Je čoraz jednoduchšie študovať techniky strojového učenia a preto progres v tomto odvetví je skutočne viditeľný. Množstvo dát a dobrá výpočtová technika majú za následok, že v takmer všetkých oblastiach sa zavádza nejaký druh umelej inteligencie. Počítače, ktoré by rozumeli informáciám by znamenali revolúciu v našich životoch. Program, ktorý by dokázal vygenerovať obraz na základe hudobného podkladu, na základe emócií a nálad, ktoré sú obsiahnuté v hudbe by bol pokrok ku umelej inteligencii, ktorá by skutočne rozumela dátam.

Proces syntézy jedného druhu informácií na iní je pre ľudí prirodzený no pre stroj je to neľahká úloha. Avšak progres v neurónových sieťach a v generatívnych algoritmoch umožňuje klasifikáciu jednej informácie a jej následnú zmenu na inú formu. Stále ale existuje množstvo prekážok v realizácií tohto problému.

To všetko nás privádza k otázke, sú dnešné neurónové siete schopné previesť hudobnú skladbu na zmysluplný obraz?
Prevod hudobnej informácie na obrazovú si vyžaduje určitý stupeň kreativity a znalostí. 
V našej práci sa budeme snažiť zodpovedať tento problém.
Budeme sa snažiť vytvoriť model, ktorý by simuloval ľudskú kreativitu.

V prvom rade je dôležité upraviť dáta, ktoré budeme analyzovať. Ide o zvukové signály, ktoré ako také sú nespracovateľné dnešnými algoritmami strojového učenia.
Je nevyhnutné aby sme tieto dáta upravili na použiteľnú formu.
Ďalším krokom je zistenie či sú počítače vôbec schopné priradenia najjednoduchšej obrazovej formy, čiže farby, k hudobným skladbám.
Úspešné splnenie tejto úlohy bude dobrým predpokladom pre vytvorenie finálneho modelu, ktorý dokáže generovať obrázky na vyššej kreatívnej úrovni.

Naša práca je preto rozdelená presne podľa týchto celkov.
Prvé kapitoly poskytnú súčasné úspechy v syntéze dát a teoretický základ pre naše riešenie.
V ďalších kapitolách postupne prejdeme naše výsledky od najjednoduchších modelov až po tie zložitejšie.
Na konci poskytneme porovnanie nami vytvorených systémov a odvodenie záverov.  
